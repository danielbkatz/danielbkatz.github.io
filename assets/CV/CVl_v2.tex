

\documentclass[12pt, oneside,]{memoir}
\usepackage{lmodern}
\usepackage{amssymb,amsmath}
\usepackage{ifxetex,ifluatex}
\usepackage{fancyhdr}
\usepackage{parskip}
\usepackage{titlesec}
\titleformat*{\section}{\Large\bfseries}

\titlespacing\section{0pt}{12pt plus 4pt minus 2pt}{0pt plus 2pt minus 2pt}
\titlespacing\subsection{0pt}{12pt plus 4pt minus 2pt}{0pt plus 2pt minus 2pt}
\titlespacing\subsubsection{0pt}{12pt plus 4pt minus 2pt}{0pt plus 2pt minus 2pt}

% \setlength\parindent{0pt} % sets indent to zero
\setlength{\parskip}{7pt}
\ifnum 0\ifxetex 1\fi\ifluatex 1\fi=0 % if pdftex
  \usepackage[T1]{fontenc}
  \usepackage[utf8]{inputenc}
\else % if luatex or xelatex
  \ifxetex
    \usepackage{mathspec}
  \else
    \usepackage{fontspec}
  \fi
  \defaultfontfeatures{Ligatures=TeX,Scale=MatchLowercase}
\fi
% use upquote if available, for straight quotes in verbatim environments

% use microtype if available
\IfFileExists{microtype.sty}{%
\usepackage{microtype}
\UseMicrotypeSet[protrusion]{basicmath} % disable protrusion for tt fonts
}{}
\usepackage[margin=1in]{geometry}
\usepackage{hyperref}
\hypersetup{unicode=true,
            pdftitle={Daniel Katz},
            pdfborder={0 0 0},
            breaklinks=true}
\urlstyle{same}  % don't use monospace font for urls
\usepackage{graphicx,grffile}
%\makeatletter
%\def\maxwidth{\ifdim\Gin@nat@width>\linewidth\linewidth\else\Gin@nat@width\fi}
%\def\maxheight{\ifdim\Gin@nat@height>\textheight\textheight\else\Gin@nat@height\fi}
\makeatother

\setlength{\emergencystretch}{3em}  % prevent overfull lines

\providecommand{\tightlist}{%
  \setlength{\itemsep}{0pt}\setlength{\parskip}{0pt}}
\setcounter{secnumdepth}{0}
% Redefines (sub)paragraphs to behave more like sections
\ifx\paragraph\undefined\else
\let\oldparagraph\paragraph
\renewcommand{\paragraph}[1]{\oldparagraph{#1}\mbox{}}
\fi
\ifx\subparagraph\undefined\else
\let\oldsubparagraph\subparagraph
\renewcommand{\subparagraph}[1]{\oldsubparagraph{#1}\mbox{}}
\fi

%%% Use protect on footnotes to avoid problems with footnotes in titles
\let\rmarkdownfootnote\footnote%
\def\footnote{\protect\rmarkdownfootnote}

%%% Change title format to be more compact
\usepackage{titling}

% Create subtitle command for use in maketitle
\providecommand{\subtitle}[1]{
  \posttitle{
    \begin{center}\large#1\end{center}
    }
}

    \settypeblocksize{9in}{6.5in}{*}
    \setlrmarginsandblock{1in}{1in}{*}
    \setulmarginsandblock{1.5in}{1in}{*}
    % Set header and footer size
    \setheadfoot{4\baselineskip}{\baselineskip}
    \checkandfixthelayout

\copypagestyle{headers}{plain}
    \makeoddhead{headers}
        %left side
        {\currentAddress}
        % center
        {{\HUGE\bfseries\name}\\ \vspace{0.5em} {\footnotesize\email \\ \phone }}
        % right side
        {\website}
    % A horizontal rule beneath the header looks nice
    \makeheadrule{headers}{\textwidth}{\normalrulethickness}
    % Activate your custom header
    %\pagestyle{headers}


 % Now supply the information to be put into the header above: This makes it easier to change
    \newcommand{\name}{Daniel Katz}
    \newcommand{\currentAddress}{UCSB \\Gevirtz Graduate School of Education}
    \newcommand{\website}{\url{https://dbkatz.com/}}
    \newcommand{\email}{katzdanb@gmail.com}
    \newcommand{\phone}{(310) 800-8354}

\pagestyle{fancy}
\fancyhf{}
\fancyhead[L]{Daniel Katz}

\begin{document}
\thispagestyle{headers}

\hypertarget{education}{%
\section{EDUCATION}\label{education}}
\begin{hangparas}{.25in}{1}

\hypertarget{university-of-california-santa-barbara}{%
\subsection{University of California, Santa
Barbara}\label{university-of-california-santa-barbara}}
\noindent\textbf{Ph.D} (2022): Education - Quantitative Research Methods and
Philosophy of Measurement\\
\hspace*{0.333em} Thesis: Unveiling uncertainties in educational measurement and psychometrics\\
\hspace*{0.333em} Committee: Andrew Maul (chair), Karen Nylund-Gibson, Diana Arya

\noindent \textbf{M.A.} (2017): Education - Quantitative Research Methods\\
\hspace*{0.333em} Thesis: Validating a multidimensional
measure of reading strategy use
\vspace{2mm}

\noindent \textbf{B.A.} (2011): Political Science, Minor in History
\end{hangparas}


\hypertarget{research-interests}{%
\section{RESEARCH INTERESTS}\label{research-interests}}
\noindent\hspace*{0.333em} The Rasch Model and Explanatory Item Response Theory Models\\
\hspace*{0.333em} Fairness and Ethics in Human Measurement\\
\hspace*{0.333em} Philosophy of Measurement and Uncertainty\\
\hspace*{0.333em} Philosophy of Language\\
\hspace*{0.333em} Literacy

\hypertarget{relevant-experience}{%
\section{RELEVANT EXPERIENCE}\label{relevant-experience}}
\begin{hangparas}{.25in}{1}

\noindent \textbf{2021 to Present}: Community Based Literacies - UCSB\\
\hspace*{0.333em}\hspace*{0.333em} Role: Measurement Specialist and Analyst\\
\hspace*{0.333em}\hspace*{0.333em} Supervisor: Diana Arya

\noindent \textbf{2021 to Present}: Microsoft (Vendor/Contractor)\\
\hspace*{0.333em}\hspace*{0.333em} Role: Measure Construction and Psychometrics for User Research

\noindent \textbf{2019-2021}: University of Florida, Virtual Learning Lab -- IES Grant R305C160004\\
\hspace*{0.333em}\hspace*{0.333em} Role: Graduate Student Psychometrics Extern -- Simulation Focus\\
\hspace*{0.333em}\hspace*{0.333em} Supervisor: Anne Corinne Manley

\noindent \textbf{2018-2021}: UCSB Center for Innovative Teaching, Research, and Learning
(CITRAL)\\
\hspace*{0.333em}\hspace*{0.333em}Role: Graduate Student Researcher in
Assessment and Evaluation\\
\hspace*{0.333em}\hspace*{0.333em}Supervisor: Linda Adler-Kassner

\noindent \textbf{Summer 2018}: New York City Department of Education\\
Assessment, Design, and Evaluation\\
\hspace*{0.333em}\hspace*{0.333em}Role: Graduate Student Psychometrics
Intern -- Applying Explanatory IRT Models\\
\hspace*{0.333em}\hspace*{0.333em}Supervisor: Ronli Diakow

\noindent \textbf{2016-2019}: California Dropout Research Project (CDRP)\\
\hspace*{0.333em}\hspace*{0.333em}Role: Graduate Student Researcher\\
\hspace*{0.333em}\hspace*{0.333em}PI: Russell Rumberger

\noindent \textbf{2016-2018}: UCSB Office of Institutional Research\\
\hspace*{0.333em}\hspace*{0.333em}Role: Graduate Student Researcher\\
\hspace*{0.333em}\hspace*{0.333em}PI: Linda Adler-Kassner, Co-Interim
Dean of UCSB Undergraduate Education
\vspace{4mm}
\end{hangparas}

\hypertarget{published-work}{%
\section{PUBLISHED WORK}\label{published-work}}
\begin{hangparas}{.25in}{1}

\textbf{Katz, D}, Huggins-Manley, A.C., Leite, W.L. (accepted). Personalized Online Learning, Test Fairness, and Educational Measurement: Considering Differential Content Exposure Prior to a High Stakes End-of-Course Exam.\emph{Applied Measurement in Education}.

Wilton, M, \textbf{Katz, D}., Clairmont, A., Gonzalez-Nino, E., Foltz, K., Christoffersen, R., (in press). Improving academic performance and retention of first-year biology students through a scalable peer-mentorship program. \emph{CBE - Life Sciences Education}.

Arya, Diana, Clairmont, A. \textbf{Katz, D} \& Maul, A. (2020) Measuring
Reading Strategy Use, \emph{Educational  Assessment}, 25:1, 5-30, DOI:
10.1080/10627197.2019.1702464

\noindent Maul, A. and \textbf{Katz, D}. (2018). Internal Validity. In B. Frey
(Editor), The SAGE Encyclopedia of Educational Research, Measurement,
and Evaluation. Thousand Oaks, CA: SAGE.

\noindent \textbf{Katz, D.} (2017, May). An Update: The Narrowing California High
School Graduation Gap between Black, Latino, and White Students. (CDRP
Statistical Brief No.24). Retrieved from
\url{http://cdrpsb.org/pubs_statbriefs.htm}

\noindent \textbf{Katz, D.} (2017, March). The Narrowing California High School
Graduation Gap between Black, Latino, and White Students. (CDRP
Statistical Brief No.~23). Retrieved from
\url{http://cdrpsb.org/pubs_statbriefs.htm}. The Narrowing California
High School Graduation Gap between Black, Latino, and White Students
\end{hangparas}
\vspace{4mm}

\hypertarget{conference-presentations}{%
\section{CONFERENCE PRESENTATIONS}\label{conference-presentations}}
\begin{hangparas}{.25in}{1}
Clairmont, A. \& \textbf{Katz, D.} (April, 2021). Using Rasch Measurement Theory for Responsive Program Evaluation. Paper presented at the American Educational Research Association Annual Meeting (AERA). To be held Virtually

\textbf{Katz, D}, Huggins-Manley, A. C., \& Leite, Walter(July, 2020). Technology-Enhanced Learning Platforms, Opportunity to Learn, and Test Fairness. Paper to be presented at the International Meeting of the Psychometric Society (IMPS). To be presented at IMPS 2021.

\textbf{Katz, D} and Clairmont A. (April, 2020). Should Psychometricians Make Claims About Test Fairness? Paper to be presented at the National Council on Measurement in Education
(NCME) (Postponed due to COVID).

\textbf{Katz, D.} and Diakow, R. (April, 2019). Using Explanatory Item
Response Theory Models to Re-Examine Fairness in Psychometrics. Paper
to be presented at the National Council on Measurement in Education
(NCME) April 4-8, 2019. Toronto.

\noindent \textbf{Katz, D.}, Nylund-Gibson, K., \& Furlong, M. (2019, April). Is One
Item Enough? Examining Affect-Laden Survey Items Using Mixture
Modelling with Distal Outcomes. Paper presented at American
Educational Research Association Annual Meeting, April 5-9, 2019.
Toronto.\\
\textbf{Best Graduate Student Paper in the AERA Survey Special Interest
Group Sig}

\noindent \textbf{Katz, D.}, Clairmont, A., Arya, D., \& Maul, A., (2018, April).
Measuring Reading Strategy Use in a Multilingual Context. Paper
presented at American Educational Research Association Annual Meeting,
April 13-17, 2018. New York.

\noindent \textbf{Katz, D.}, Clairmont, A., Arya, D. \& Maul, A. (2018, April).
Measuring Reading Strategy Use in a Multidimensional, Multilingual
Context. Paper presented at the International Objective Measurement
Workshop (IOMW), April 10-12, 2018. New York.
\end{hangparas}
\vspace{4mm}

\hypertarget{select-in-progress-work}{%
\section{SELECT IN PROGRESS WORK}\label{select-in-progress-work}}
\begin{hangparas}{.25in}{1}

Katz, D., Maul A., Clairmont, A (2021) Tools from philosophy of language
for increasing transparency in psychological measurement"

\noindent Clairmont, A. \& \textbf{Katz D.} (in preperation) Using Rasch
Measurement Theory for program evaluation: A methods note. To be
submitted to \emph{American Journal of Evaluation}

\noindent \textbf{Katz D.} Reconsidering fairness in educational assessment: using
tools from ethics, causal inference, and psychometrics

\noindent \textbf{Katz D.} Contemplating the measurement of subjective well-being and other constructs in positive psychology using mixture modelling approaches (with an emphasis on educational settings)
\end{hangparas}
\vspace{4mm}

\hypertarget{teaching-experience}{%
\section{TEACHING EXPERIENCE}\label{teaching-experience}}

\noindent\textbf{Instructor}\\
\noindent One Day Workshops\\
\hspace*{0.333em}\hspace*{0.333em}UCSB Methods U - An Introduction to R: Data Cleaning, Wrangling, and Visualizing\\
\hspace*{0.333em}\hspace*{0.333em}Implementing Explanatory IRT via the R package, TAM
\vspace{4mm}

\noindent \textbf{Teaching Assistant}\\
\hspace*{0.333em}\hspace*{0.333em}Fall 2020: Education 217C, Constructing Measures\\
\hspace*{0.333em}\hspace*{0.333em}Spring 2018: Education 214C, Linear Models for Data Analysis (and categorical data analysis)\\
\hspace*{0.333em}\hspace*{0.333em}Winter 2018: Education 214B,Inferential Statistics\\
\hspace*{0.333em}\hspace*{0.333em}Fall 2017: Education 214A, Introductory Statistics
\vspace{4mm}

\hypertarget{organization-committee-memberships}{%
\section{ORGANIZATION \& COMMITTEE
MEMBERSHIPS}\label{organization-committee-memberships}}

\noindent Graduate Student Representative - NCME Committee on Informing Assessment Policy and Practice\\
American Educational Research Association\\
National Council on Measurement in Education\\
Society for the Improvement of Psychological Sciences
\vspace{4mm}

\hypertarget{statistical-software}{%
\section{STATISTICAL SOFTWARE}\label{statistical-software}}
\noindent R\\
\hspace*{0.333em} Commonly used stats packages: TAM, mirt, lme4, Lavaan, plm\\
\hspace*{0.333em} Stan, brms, rstan (Basic Knowledge)\\
\hspace*{0.333em} Other packages: tidyverse packages, Rmarkdown, xaringan\\
\noindent Mplus\\
\noindent Stata (Basic Knowledge)\\
Tableau (Basic Knowledge)\\
Git and Gitub (Basic Knowledge)\\
Dedoose (Beginner)

\end{document}
