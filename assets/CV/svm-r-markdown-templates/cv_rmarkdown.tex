\documentclass[]{article}
\usepackage{lmodern}
\usepackage{amssymb,amsmath}
\usepackage{ifxetex,ifluatex}
\usepackage{fixltx2e} % provides \textsubscript
\ifnum 0\ifxetex 1\fi\ifluatex 1\fi=0 % if pdftex
  \usepackage[T1]{fontenc}
  \usepackage[utf8]{inputenc}
\else % if luatex or xelatex
  \ifxetex
    \usepackage{mathspec}
  \else
    \usepackage{fontspec}
  \fi
  \defaultfontfeatures{Ligatures=TeX,Scale=MatchLowercase}
\fi
% use upquote if available, for straight quotes in verbatim environments
\IfFileExists{upquote.sty}{\usepackage{upquote}}{}
% use microtype if available
\IfFileExists{microtype.sty}{%
\usepackage{microtype}
\UseMicrotypeSet[protrusion]{basicmath} % disable protrusion for tt fonts
}{}




\setlength{\emergencystretch}{3em}  % prevent overfull lines
\providecommand{\tightlist}{%
  \setlength{\itemsep}{0pt}\setlength{\parskip}{0pt}}
\setcounter{secnumdepth}{0}
% Redefines (sub)paragraphs to behave more like sections
\ifx\paragraph\undefined\else
\let\oldparagraph\paragraph
\renewcommand{\paragraph}[1]{\oldparagraph{#1}\mbox{}}
\fi
\ifx\subparagraph\undefined\else
\let\oldsubparagraph\subparagraph
\renewcommand{\subparagraph}[1]{\oldsubparagraph{#1}\mbox{}}
\fi

% Now begins the stuff that I added.
% ----------------------------------

% Custom section fonts
\usepackage{sectsty}
\sectionfont{\rmfamily\mdseries\large\bf}
\subsectionfont{\rmfamily\mdseries\normalsize\scshape}


% Make lists without bullets
\renewenvironment{itemize}{
  \begin{list}{}{
    \setlength{\leftmargin}{1.5em}
  }
}{
  \end{list}
}


% Make parskips rather than indent with lists.
\usepackage{parskip}
\usepackage{titlesec}
\titlespacing\section{0pt}{12pt plus 4pt minus 2pt}{4pt plus 2pt minus 2pt}
\titlespacing\subsection{0pt}{12pt plus 4pt minus 2pt}{4pt plus 2pt minus 2pt}

% Use fontawesome. Note: you'll need TeXLive 2015. Update.


% Fancyhdr, as I tend to do with these personal documents.
\usepackage{fancyhdr,lastpage}
\pagestyle{fancy}
\renewcommand{\headrulewidth}{0.0pt}
\renewcommand{\footrulewidth}{0.0pt}
\lhead{}
\chead{}
\rhead{}
\lfoot{
\cfoot{\scriptsize   -  }}
\rfoot{\scriptsize \thepage/{\hypersetup{linkcolor=black}\pageref{LastPage}}}

% Always load hyperref last.
\usepackage{hyperref}

\hypersetup{unicode=true,
            colorlinks=true,
            linkcolor=Maroon,
            citecolor=Blue,
            urlcolor=Blue,
            breaklinks=true, bookmarks=true}
\urlstyle{same}  % don't use monospace font for urls

% Make AP style (kinda) dates for the updated/today field

\usepackage{datetime}
\newdateformat{apstylekinda}{%
  \shortmonthname[\THEMONTH]. \THEDAY, \THEYEAR}

\begin{document}


\centerline{\huge \bf }

\vspace{2 mm}

\hrule

\vspace{2 mm}



\moveleft.5\hoffset\centerline{        }

\vspace{2 mm}

\hrule


\begin{center}\rule{0.5\linewidth}{\linethickness}\end{center}

output: pdf\_document: latex\_engine: pdflatex template:
C:/Users/katzd/Desktop/Github/danielbkatz.github.io/assets/CV/svm-r-markdown-templates/svm-latex-cv.tex

geometry: margin=1in

title: ``CV'' author: ``Daniel Katz'' jobtitle: ``Graduate Student''
address: ``387 Cannon Green Dr., Goleta, CA, 93117'' fontawesome: yes
email: \href{mailto:dkatz@ucsb.edu}{\nolinkurl{dkatz@ucsb.edu}} github:
danielbkatz phone: ``310-800-8354'' web: dbkatz.com updated: no

fontfamily: mathpazo fontfamilyoptions: sc, osf fontsize: 11pt
linkcolor: blue urlcolor: blue ---

\hypertarget{education}{%
\subsection{EDUCATION}\label{education}}

\textbf{PhD} (2021): Education - Research Methods and Philosophy of
Measurement\\
University, of California, Santa Barbara\\
Advisor: Andrew Maul\\
\textbf{M.A.} (2017) Education (Quantitative Research Methods)
\textbf{Thesis:} Validating a Multidimensional Measure of Reading
Strategy Use with the Rasch Model University of California, Santa
Barbara\\
Committee: Andrew Maul, Karen Nylund-Gibson, Diana Arya

\textbf{BA} (2011), Political Science, Minor in History\\
University of California, Santa Barbara

\hypertarget{research-interests}{%
\subsection{\#\# RESEARCH INTERESTS}\label{research-interests}}

The Rasch Model and Item Response Theory\\
Measurement Invariance\\
Explanatory Item Response Models\\
Philosophy of Measurement and Probability\\
Casual Inference\\
Measuring student literacy and other non-academic abilities

\hypertarget{relevant-experience}{%
\subsection{\#\# RELEVANT EXPERIENCE}\label{relevant-experience}}

\textbf{University of Florida, Virtual Learning Lab -- IES Funded Grant}
(2019-Present) * Supervisor: Anne Corinne Manley * Role: Graduate
Student Extern/Data Analyst -- IRT/Simulation focus

\textbf{UCSB Center for Innovative Teaching, Research, and Learning
(CITRAL)} (2018-Present) * Supervisor: Linda Adler-Kassner * Role:
Graduate Student Researcher in Assessment and Evaluation

\textbf{New York City Department of Education Assessment, Design, and
Evaluation Team} (Summer 2018) * Supervisor: Ronli Diakow * Role:
Graduate Student Psychometrics Intern -- Applying IRT and Explanatory
IRT Models

\textbf{UCSB California Dropout Research Project (CDRP) and Get Focused
Stay Focused Evaluations} (2016-2019) * PI: Russell Rumberger * Role:
Graduate Student Researcher (data analyst)

\textbf{UCSB General Education Assessment Project, Office of
Institutional Research and Assessment Research Group} (2016-2018) * PI:
Linda Adler-Kassner, Co-Interim Dean of UCSB Undergraduate Education *
Role: Graduate Student Researcher

\hypertarget{published-work}{%
\subsection{\#\# PUBLISHED WORK}\label{published-work}}

Arya, Diana, Clairmont, A. \textbf{Katz, Daniel} \& Maul, A. (2020)
Measuring Reading Strategy Use, Educational Assessment, 25:1, 5-30, DOI:
10.1080/10627197.2019.1702464

Maul, A. and \textbf{Katz, D}. (2018). Internal Validity. In B. Frey
(Editor), The SAGE Encyclopedia of Educational Research, Measurement,
and Evaluation. Thousand Oaks, CA: SAGE.

\textbf{Katz, D.} (2017, May). An Update: The Narrowing California High
School Graduation Gap between Black, Latino, and White Students. (CDRP
Statistical Brief No.~24). Retrieved from
\url{http://cdrpsb.org/pubs_statbriefs.htm}

\textbf{Katz, D.} (2017, March). The Narrowing California High School
Graduation Gap between Black, Latino, and White Students. (CDRP
Statistical Brief No.~23). Retrieved from
\url{http://cdrpsb.org/pubs_statbriefs.htm}. The Narrowing California
High School Graduation Gap between Black, Latino, and White Students

\hypertarget{conference-presentations}{%
\subsection{\#\# CONFERENCE
PRESENTATIONS}\label{conference-presentations}}

\textbf{Katz, D.} and Diakow, R. (2019, April). Using Explanatory Item
Response Theory Models to Re-Examine Fairness in Psychometrics. Paper to
be presented at the National Council on Measurement in Education (NCME)
April 4-8, 2019. Toronto.

\textbf{Katz, D.}v, Nylund-Gibson, K., Furlong, M. (2019, April). Is One
Item Enough? Examining Affect-Laden Survey Items Using Mixture Modelling
with Distal Outcomes. Paper presented at American Educational Research
Association Annual Meeting, April 5-9, 2019. Toronto.\\
\textbf{Best Graduate Student Paper in the AERA Survey Special Interest
Group Sig}

\textbf{Katz, D.}, Clairmont, A., Arya, D., \& Maul, A., (2018, April).
Measuring Reading Strategy Use in a Multilingual Context. Paper
presented at American Educational Research Association Annual Meeting,
April 13-17, 2018. New York.

\textbf{Katz, D.}, Clairmont, A., Arya, D. \& Maul, A. (2018, April).
Measuring Reading Strategy Use in a Multidimensional, Multilingual
Context. Paper presented at the International Objective Measurement
Workshop (IOMW), April 10-12, 2018. New York.

\hypertarget{select-in-progress-research-projects}{%
\subsection{\#\# SELECT IN PROGRESS RESEARCH
PROJECTS}\label{select-in-progress-research-projects}}

Clairmont, A. \& Katz D. (in preperation) Using Rasch Measurement Theory
for Program Evaluation: A Methods Note. To be submitted to
\emph{American Journal of Evaluation}

Reconsidering fairness in educational assessment: using tools from
ethics, causal inference, and psychometrics

Using Rasch Measurement Theory for program evaluation

Measuring subjective well-being and other constructs in positive
psychology using mixture modelling approaches (with an emphasis on
educational settings).

\hypertarget{teaching-experience}{%
\subsection{\#\# TEACHING EXPERIENCE}\label{teaching-experience}}

\textbf{Instructor}\\
+ One Day Workshop: UCSB Methods U - An Introduction to R: Data
Cleaning, Wrangling, and Visualizing + Implementing Explanatory IRT via
the R package, TAM -- Course Materials\\

\textbf{Teaching Assistant} + Education 214C, Linear Models for Data
Analysis (and categorical data analysis), Spring 2018 + Education 214B,
Inferential Statistics, Winter 2018 + Education 214A, Introductory
Statistics, Fall 2017

\hypertarget{organization-committee-memberships}{%
\subsection{\#\# ORGANIZATION \& COMMITTEE
MEMBERSHIPS}\label{organization-committee-memberships}}

\begin{itemize}
\tightlist
\item
  Graduate Student Representative -- NCME Committee on informing
  assessment policy and practice
\item
  American Educational Research Association
\item
  National Council on Measurement in Education
\item
  Society for the Improvement of Psychological Sciences
\end{itemize}

\hypertarget{statistical-software-knowledge}{%
\subsection{\#\# STATISTICAL SOFTWARE
KNOWLEDGE}\label{statistical-software-knowledge}}

\begin{itemize}
\tightlist
\item
  R

  \begin{itemize}
  \tightlist
  \item
    Commonly used stats packages: TAM, mirt, lme4, Lavaan, plm
  \item
    Other packages: tidyverse, Rmarkdown, xaringan
  \end{itemize}
\item
  Mplus
\item
  Stata (Basic Knowledge)\\
\item
  Tableau (Basic Knowledge)\\
\item
  Git and Gitub (Basic Knowledge)\\
\item
  Stan/rstan (Beginner)\\
\item
  Dedoose (Beginner)
\end{itemize}

\end{document}
