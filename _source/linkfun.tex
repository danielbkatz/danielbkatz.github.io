% Options for packages loaded elsewhere
\PassOptionsToPackage{unicode=true}{hyperref}
\PassOptionsToPackage{hyphens}{url}
%
\documentclass[
]{article}
\usepackage{lmodern}
\usepackage{amssymb,amsmath}
\usepackage{ifxetex,ifluatex}
\ifnum 0\ifxetex 1\fi\ifluatex 1\fi=0 % if pdftex
  \usepackage[T1]{fontenc}
  \usepackage[utf8]{inputenc}
  \usepackage{textcomp} % provides euro and other symbols
\else % if luatex or xelatex
  \usepackage{unicode-math}
  \defaultfontfeatures{Scale=MatchLowercase}
  \defaultfontfeatures[\rmfamily]{Ligatures=TeX,Scale=1}
\fi
% Use upquote if available, for straight quotes in verbatim environments
\IfFileExists{upquote.sty}{\usepackage{upquote}}{}
\IfFileExists{microtype.sty}{% use microtype if available
  \usepackage[]{microtype}
  \UseMicrotypeSet[protrusion]{basicmath} % disable protrusion for tt fonts
}{}
\makeatletter
\@ifundefined{KOMAClassName}{% if non-KOMA class
  \IfFileExists{parskip.sty}{%
    \usepackage{parskip}
  }{% else
    \setlength{\parindent}{0pt}
    \setlength{\parskip}{6pt plus 2pt minus 1pt}}
}{% if KOMA class
  \KOMAoptions{parskip=half}}
\makeatother
\usepackage{xcolor}
\IfFileExists{xurl.sty}{\usepackage{xurl}}{} % add URL line breaks if available
\IfFileExists{bookmark.sty}{\usepackage{bookmark}}{\usepackage{hyperref}}
\hypersetup{
  pdftitle={Link Functions: Where do they come from?},
  pdfauthor={Danny},
  hidelinks,
}
\urlstyle{same} % disable monospaced font for URLs
\usepackage[margin=1in]{geometry}
\usepackage{graphicx,grffile}
\makeatletter
\def\maxwidth{\ifdim\Gin@nat@width>\linewidth\linewidth\else\Gin@nat@width\fi}
\def\maxheight{\ifdim\Gin@nat@height>\textheight\textheight\else\Gin@nat@height\fi}
\makeatother
% Scale images if necessary, so that they will not overflow the page
% margins by default, and it is still possible to overwrite the defaults
% using explicit options in \includegraphics[width, height, ...]{}
\setkeys{Gin}{width=\maxwidth,height=\maxheight,keepaspectratio}
\setlength{\emergencystretch}{3em} % prevent overfull lines
\providecommand{\tightlist}{%
  \setlength{\itemsep}{0pt}\setlength{\parskip}{0pt}}
\setcounter{secnumdepth}{-\maxdimen} % remove section numbering
% Redefines (sub)paragraphs to behave more like sections
\ifx\paragraph\undefined\else
  \let\oldparagraph\paragraph
  \renewcommand{\paragraph}[1]{\oldparagraph{#1}\mbox{}}
\fi
\ifx\subparagraph\undefined\else
  \let\oldsubparagraph\subparagraph
  \renewcommand{\subparagraph}[1]{\oldsubparagraph{#1}\mbox{}}
\fi

% Set default figure placement to htbp
\makeatletter
\def\fps@figure{htbp}
\makeatother


\title{Link Functions: Where do they come from?}
\author{Danny}
\date{3/30/2020}

\begin{document}
\maketitle

As a student, when presented with the concept of link functions -
especially in a more applied statistics course - I was always left
wanting a bit more. The standard way a social science student usually
learns about link functions is through the introduction of logistic
regression. In this post, I will give an earlier version of myself what
I always wanted - a step-by-step break down of how the link function is
derived algebraically. In a future post, I'll expand on the conceptual
aspects of link functions - introducing roughl, how probability enters
are generalized linear models which is paramount for understanding link
functions.

Of note, we'll be dealing with link functions in only super specific
scenarios. First, we'll be dealing with canonical link functions, only.
For instance, the logit is the cannonical link to the binomial whereas
the probit is not. Roughly speaking, this means that the logit link can
be directly derived from the binomial whereas the probit link cannot.

For this post, we'll be using primarily algebra - there will be a hint
of calculus. If this isn't your cup of tea, perhaps the next post will
be more interesting. But, I try to explain all the steps below in a way
that I hope is clear, including the ``rules'' I'm using through each
step - so give it a shot! I think a little effort doing algebra can
serve most of us students in social science more than we realize.

\hypertarget{preliminaries}{%
\subsection{Preliminaries:}\label{preliminaries}}

I'll refer to these rules in the post below.

\begin{enumerate}
\def\labelenumi{\arabic{enumi}.}
\tightlist
\item
  When I use \[Log\] I mean the natural log, or \(ln\) . This is the
  logarithm with base \(e\) such that \[log_e(1) = 0\] because
  \[e^0 =1\] .
\item
  log(a) + log(b) = \[log(a*b)\] .
\item
  log(a)-log(b) = \[log(\frac{a}{b})\] .
\item
  \[log(a)^x = xlog(a)\] .
\end{enumerate}

\hypertarget{exponential-family-of-probability-functions}{%
\subsection{Exponential Family of Probability
Functions}\label{exponential-family-of-probability-functions}}

This is where the bulk of the hard work will be and it's all going to be
algebra. It turns out that probability density functions like the normal
distribution or discrete probability distributions like the binomial
(called probability mass functions) are of a family of distributions
called the exponential family of distributions. In the case of the
binomial, it is in the exponential family only when there are a fixed
number of trials, but that's not important.

What's special about this is that members of this family can be written
in the common form,\\
\begin{equation}
  $$f_X(x|\theta) = h(x)exp[\eta(\theta)*T(x) - A(\theta)]$$

\end{equation}

For whatever reason, I like an equivalent form of this distribution a
bit better:

\[f(x, \theta, \phi) = exp[\frac{y\theta-b(\theta)}{a(\phi)} + c(y, \phi)]\]

One notes that this looks nothing like the binomial distribution in its
generalized linear model form (glm)

\[f(y) = {n \choose y}p^y(1-p)^{n-y}\] ..

So we'll have to transform it into the form above.

\end{document}
