% Options for packages loaded elsewhere
<<<<<<< HEAD
\PassOptionsToPackage{unicode}{hyperref}
=======
\PassOptionsToPackage{unicode=true}{hyperref}
>>>>>>> 19bdb9ffa4f78b31e460bde0aaae9e85ef799363
\PassOptionsToPackage{hyphens}{url}
%
\documentclass[
]{article}
<<<<<<< HEAD
\usepackage{amsmath,amssymb}
\usepackage{lmodern}
=======
\usepackage{lmodern}
\usepackage{amssymb,amsmath}
>>>>>>> 19bdb9ffa4f78b31e460bde0aaae9e85ef799363
\usepackage{ifxetex,ifluatex}
\ifnum 0\ifxetex 1\fi\ifluatex 1\fi=0 % if pdftex
  \usepackage[T1]{fontenc}
  \usepackage[utf8]{inputenc}
<<<<<<< HEAD
  \usepackage{textcomp} % provide euro and other symbols
\else % if luatex or xetex
=======
  \usepackage{textcomp} % provides euro and other symbols
\else % if luatex or xelatex
>>>>>>> 19bdb9ffa4f78b31e460bde0aaae9e85ef799363
  \usepackage{unicode-math}
  \defaultfontfeatures{Scale=MatchLowercase}
  \defaultfontfeatures[\rmfamily]{Ligatures=TeX,Scale=1}
\fi
% Use upquote if available, for straight quotes in verbatim environments
\IfFileExists{upquote.sty}{\usepackage{upquote}}{}
\IfFileExists{microtype.sty}{% use microtype if available
  \usepackage[]{microtype}
  \UseMicrotypeSet[protrusion]{basicmath} % disable protrusion for tt fonts
}{}
\makeatletter
\@ifundefined{KOMAClassName}{% if non-KOMA class
  \IfFileExists{parskip.sty}{%
    \usepackage{parskip}
  }{% else
    \setlength{\parindent}{0pt}
    \setlength{\parskip}{6pt plus 2pt minus 1pt}}
}{% if KOMA class
  \KOMAoptions{parskip=half}}
\makeatother
\usepackage{xcolor}
\IfFileExists{xurl.sty}{\usepackage{xurl}}{} % add URL line breaks if available
\IfFileExists{bookmark.sty}{\usepackage{bookmark}}{\usepackage{hyperref}}
\hypersetup{
  pdftitle={Link Functions: Where do they come from?},
  pdfauthor={Danny},
  hidelinks,
<<<<<<< HEAD
  pdfcreator={LaTeX via pandoc}}
\urlstyle{same} % disable monospaced font for URLs
\usepackage[margin=1in]{geometry}
\usepackage{graphicx}
=======
}
\urlstyle{same} % disable monospaced font for URLs
\usepackage[margin=1in]{geometry}
\usepackage{graphicx,grffile}
>>>>>>> 19bdb9ffa4f78b31e460bde0aaae9e85ef799363
\makeatletter
\def\maxwidth{\ifdim\Gin@nat@width>\linewidth\linewidth\else\Gin@nat@width\fi}
\def\maxheight{\ifdim\Gin@nat@height>\textheight\textheight\else\Gin@nat@height\fi}
\makeatother
% Scale images if necessary, so that they will not overflow the page
% margins by default, and it is still possible to overwrite the defaults
% using explicit options in \includegraphics[width, height, ...]{}
\setkeys{Gin}{width=\maxwidth,height=\maxheight,keepaspectratio}
<<<<<<< HEAD
% Set default figure placement to htbp
\makeatletter
\def\fps@figure{htbp}
\makeatother
=======
>>>>>>> 19bdb9ffa4f78b31e460bde0aaae9e85ef799363
\setlength{\emergencystretch}{3em} % prevent overfull lines
\providecommand{\tightlist}{%
  \setlength{\itemsep}{0pt}\setlength{\parskip}{0pt}}
\setcounter{secnumdepth}{-\maxdimen} % remove section numbering
<<<<<<< HEAD
\ifluatex
  \usepackage{selnolig}  % disable illegal ligatures
\fi
=======
% Redefines (sub)paragraphs to behave more like sections
\ifx\paragraph\undefined\else
  \let\oldparagraph\paragraph
  \renewcommand{\paragraph}[1]{\oldparagraph{#1}\mbox{}}
\fi
\ifx\subparagraph\undefined\else
  \let\oldsubparagraph\subparagraph
  \renewcommand{\subparagraph}[1]{\oldsubparagraph{#1}\mbox{}}
\fi

% Set default figure placement to htbp
\makeatletter
\def\fps@figure{htbp}
\makeatother

>>>>>>> 19bdb9ffa4f78b31e460bde0aaae9e85ef799363

\title{Link Functions: Where do they come from?}
\author{Danny}
\date{3/30/2020}

\begin{document}
\maketitle

<<<<<<< HEAD
As a student presented with the concept of link functions I was always
left wanting a bit more. The standard way a social science student
usually learns about link functions is through the introduction of
logistic regression. In this note, I will attempt to provide an earlier
version of myself what I always wanted - a step-by-step break down of
how the logit link function is derived.

Of note, we'll be dealing with link functions in specific scenarios.
First, we'll be dealing with the canonical link function, only. For
instance, the logit is the cannonical link to the binomial whereas the
probit is not. Roughly speaking, this means that the logit link can be
directly derived from the binomial whereas the probit link cannot.
Ironically, early on, probit regression was more widely used than
logistic regression.

We'll be using a lot of algebra - there will be a hint of calculus. If
this isn't your cup of tea, perhaps just try to read the step. But, I
try to explain all the steps below in a way that I hope is clear,
including the ``rules'' I'm using through each step - so give it a shot!
Usually mathematical derivations skip steps which makes them actually
look less daunting. I tried to show every single step to make it
obvious. I think a little effort doing algebra can serve most of us
students in social science more than we realize.
=======
As a student, when presented with the concept of link functions -
especially in a more applied statistics course - I was always left
wanting a bit more. The standard way a social science student usually
learns about link functions is through the introduction of logistic
regression. In this post, I will give an earlier version of myself what
I always wanted - a step-by-step break down of how the link function is
derived algebraically. In a future post, I'll expand on the conceptual
aspects of link functions - introducing roughl, how probability enters
are generalized linear models which is paramount for understanding link
functions.

Of note, we'll be dealing with link functions in only super specific
scenarios. First, we'll be dealing with canonical link functions, only.
For instance, the logit is the cannonical link to the binomial whereas
the probit is not. Roughly speaking, this means that the logit link can
be directly derived from the binomial whereas the probit link cannot.

For this post, we'll be using primarily algebra - there will be a hint
of calculus. If this isn't your cup of tea, perhaps the next post will
be more interesting. But, I try to explain all the steps below in a way
that I hope is clear, including the ``rules'' I'm using through each
step - so give it a shot! I think a little effort doing algebra can
serve most of us students in social science more than we realize.
>>>>>>> 19bdb9ffa4f78b31e460bde0aaae9e85ef799363

\hypertarget{preliminaries}{%
\subsection{Preliminaries:}\label{preliminaries}}

I'll refer to these rules in the post below.

\begin{enumerate}
\def\labelenumi{\arabic{enumi}.}
\tightlist
\item
<<<<<<< HEAD
  When I use \(log\) I mean the natural log, or \(ln\) . This is the
  logarithm with base \(e\) such that \(log_e(1) = 0\) because
  \(e^0 =1\) .
\item
  \(log(a) + log(b) = log(a*b)\) .
\item
  \(log(a)-log(b) = log(\frac{a}{b})\) .
\item
  \(log(a)^x = xlog(a)\) .
=======
  When I use \[Log\] I mean the natural log, or \(ln\) . This is the
  logarithm with base \(e\) such that \[log_e(1) = 0\] because
  \[e^0 =1\] .
\item
  log(a) + log(b) = \[log(a*b)\] .
\item
  log(a)-log(b) = \[log(\frac{a}{b})\] .
\item
  \[log(a)^x = xlog(a)\] .
>>>>>>> 19bdb9ffa4f78b31e460bde0aaae9e85ef799363
\end{enumerate}

\hypertarget{exponential-family-of-probability-functions}{%
\subsection{Exponential Family of Probability
Functions}\label{exponential-family-of-probability-functions}}

<<<<<<< HEAD
A generalized linear model's link function takes the ``structural'' or
linear component of the model and links it to the outcome, the
expectation of y given x (sometimes, this is written, y-hat) but you can
also think of it as \(\mu\). To expand, even when you have a standard
linear regression of the form,

\[y_i = \alpha + \beta_1X_{1i }+ \epsilon_i\]\\
y has some conditional distribution, an error distribution. In other
words, we need to have some way to go from the structural portion of the
model, \[\alpha + \beta_1X_1\], which is just a straight line, and link
it to its outcome \(y\) and have some probability density function for
describing or modeling the data, since we are not working with
deterministic data, afterall. In this case, we have an \(\epsilon\) term
that ``adds'' noise and we assume that it's normally distibuted for each
value of X. So Y, for a given value of x, has a normal distribution. In
other words, \[E(Y|X) = \mathcal{N(\mu=\alpha + B_1X_1, \sigma^2)}\] or
in matrix notation:\\
\[E(Y|X) = \mathcal{N(\mu=XB, \sigma^2I)}\] where I is the identity
matrix. So, in this case, the link function is the idenity - or, said
another way, the identify link takes \(E[Y|X]\) and outputs the same
thing - it doesn't change it - \(g(E[Y|X]) = E[Y|X]\).

When the outcome can only take a value of 1 or 0, the normal
distribution is not a grear distribution to use, really - it best models
data that are continuous (among many other properties). So, the identity
link won't do. We need to see if we can use some other function to
transform the expectation such that we can still use a linear model to
model the data.

Welcome to link functions for the generalized linear model.

\hypertarget{family-of-exponential-distributions}{%
\subsection{Family of Exponential
Distributions}\label{family-of-exponential-distributions}}

This is where the bulk of the hard work will be. It turns out that
probability density functions (pdf) like the normal distribution or
discrete probability distributions (probability mass functions, or, pmf)
like the binomial are of a family of distributions called the
exponential family of distributions. In the case of the binomial, it is
in the exponential family only when there are a fixed number of trials.

What's special about this is that members of this family can be written
in the general form,

\begin{align}
\tag{A}
f_X(x|\theta) = h(x)exp[\eta(\theta)*T(x) - A(\theta)]
\end{align}
=======
This is where the bulk of the hard work will be and it's all going to be
algebra. It turns out that probability density functions like the normal
distribution or discrete probability distributions like the binomial
(called probability mass functions) are of a family of distributions
called the exponential family of distributions. In the case of the
binomial, it is in the exponential family only when there are a fixed
number of trials, but that's not important.

What's special about this is that members of this family can be written
in the common form,\\
\begin{equation}
  $$f_X(x|\theta) = h(x)exp[\eta(\theta)*T(x) - A(\theta)]$$

\end{equation}
>>>>>>> 19bdb9ffa4f78b31e460bde0aaae9e85ef799363

For whatever reason, I like an equivalent form of this distribution a
bit better:

<<<<<<< HEAD
\begin{align}
\tag{B}
f(x, \theta, \phi) = exp\biggl[\frac{y\theta-b(\theta)}{a(\phi)} + c(y, \phi)\biggr] 
\end{align}

Here, y is your outcome (or number of successes in the case of a
binomial), \(\theta\) is a parameter, \(b(\theta)\) is a function of
\(\theta\), \(a(\phi)\) is a function of parameter \(\phi\), and
\texttt{c()} is a function or some set of functions of data and
\(\phi\).

\hypertarget{the-logit-link-and-logistic-regression}{%
\subsection{The logit link and logistic
regression}\label{the-logit-link-and-logistic-regression}}

The binomial distribution looks like:

\begin{align}
\tag{C}
f(y) = {n \choose y}p^y(1-p)^{n-y}
\end{align}

This looks nothing like \texttt{equation\ B}. Remember from
\texttt{equation\ A,} above, that a \texttt{GLM} gives us the
expectation of y given x. In linear regression, the expectation of y
given an x value is simply, \(\hat{y} = \alpha + B1_X1\)

We now have a dichotomous outcome which simply can't be normally
distributed given each value of \texttt{X} (aka, it can't have normally
distributed errors). This will make more sense if we can link the
structural portion of the model to the parameter of the binomial, which
is, in the case of 1 trial, simply a bernoulli distribution with
expecation, \(p\), or, generally, with \(n\) trials, \(n*p\) where \(p\)
is the probability of success for a trial. In this way, we're saying our
outcome is the result of bernoulli trials. Regardless, we have to
transform to find the link function.

\hypertarget{transforming-the-binomial-into-exponential-form}{%
\subsection{Transforming the Binomial into Exponential
Form}\label{transforming-the-binomial-into-exponential-form}}

\hypertarget{steps-1-4}{%
\subsubsection{Steps 1-4}\label{steps-1-4}}

\begin{enumerate}
\def\labelenumi{\arabic{enumi}.}
\tightlist
\item
  The first step is taking the log of both sides of
  \texttt{equation\ C}, because, from \textbf{preliminary point 2}, this
  will take the multiplication and turn it into addition which we need
  to do to get closer to \texttt{B}.
\item
  Using \textbf{prelimiary point 4}, we factor and rearrange, and place
  the exponents in front of the log fuctions (for multiplication).
\item
  To get rid of the log on the left side, we exponentiate with \(e\)
  both sides (using \(exp(a)\) to mean \(e^a\)).
\item
  Re-write so f(y) back to normal. Proceed to step 5. The algebra is
  below.
\end{enumerate}

\begin{align}
\tag{1}
log(f(y)) = log\biggl[{n\choose y}\biggr]+ log(p)^y + log(1-p)^{n-y}
\end{align}

\begin{align}
\tag{2}
log(f(y)) = log\biggl[{n\choose y}\biggr]+ ylog(p) + (n-y)log(1-p) .
\end{align}

\begin{align}
\tag{3}
exp(log(f(y))) = exp\biggl[log\biggl({n\choose y}\biggr)+ ylog(p) + (n-y)log(1-p)\biggr]
\end{align}

\begin{align}
\tag{4}
f(y) = exp\biggl[log({n\choose y})+ ylog(p) + (n-y)log(1-p)\biggr]
\end{align}

\hypertarget{steps-5---7}{%
\subsubsection{Steps 5 - 7}\label{steps-5---7}}

To save on typing, we'll only write the term inside the
\texttt{exp{[}\ {]}} term now.

\begin{enumerate}
\def\labelenumi{\arabic{enumi}.}
\setcounter{enumi}{4}
\item
  We'll expand out \((n-y)(log(1-p))\) in step 5.
\item
  Rewrite with step 5 expansion inserted back in. To make things
  simpler, we'll group our outcome, y, our parameter of interest p, and
  terms with the full data in it (n).
\item
  From \textbf{preliminary 3}, since first two terms have the same log
  base and multiplicative constant we can regroup such that
  \(ylog(p) - ylog(1-p)\) becomes \(y\biggl[log(\frac{p}{1-p})\biggr]\).
\end{enumerate}

\begin{align}
\tag{5}
nlog(1-p) - ylog(1-p)
\end{align}

\begin{align}
\tag{6}
=ylog(p) - ylog(1-p) + nlog(1-p) + log\biggl({n\choose y}\biggr)
\end{align}

\begin{align}
\tag{7}
=y\biggl[log(\frac{p}{1-p})\biggr] + nlog(1-p) + log{n\choose y}
\end{align}

\hypertarget{steps-8---10}{%
\subsection{Steps 8 - 10}\label{steps-8---10}}

This is starting to look just like we want, and if you've used logistic
regression, there should be some familiar terms.

Remember that our exponential form from equation B:

\begin{align}
\tag{B}
f(x, \theta, \phi) = exp\biggl[\frac{y\theta-b(\theta)}{a(\phi)} + c(y, \phi)\biggr] 
\end{align}

It appears the \(y\theta\) portion matches up with:\\
\(y\biggl[log(\frac{p}{1-p})\biggr]\)

\begin{align}
\tag{8}
\theta = \biggl[log(\frac{p}{1-p})\biggr]
\end{align}

From equation 2, we see that the second term is also a function of
theta, so we need to rewrite \(nlog(1-p)\) in terms of \(\theta\). So,
the easiest way to do this is to solve for \(\theta\) in terms of p.

Exponentiate both sides of \texttt{8} then simply solve for \(\theta\)
by: A. multiplying both sides by \(1-p\), expanding/multiplying out the
left side, adding \(exp(\theta)*p\) to both sides

B. factoring out \(p\), the common term from the right side,

C. Divide both sides by \((1+exp(\theta))\)

\[1 - p = \frac{1}{1 + exp(\theta)} = (1+exp(\theta))^{-1}\]

\begin{align}
\tag{9a}
exp(\theta) = \frac{p}{1-p} \\
exp(\theta)(1-p) = p \\
exp(\theta)-exp(\theta)*(p) = p \\
exp(\theta) = p + exp(\theta)*p 
\end{align}

Factor:

\begin{align}
\tag{9b}
exp(\theta) = p(1 + exp(\theta)) 
\end{align}

\begin{align}
\tag{9c}
\frac{exp(\theta)}{1+exp(\theta)} = p \\
1 = \frac{1 + exp(\theta)}{1+exp(\theta)}
\end{align}

\begin{align}
\tag{9d}
1 - p = \frac{1}{1 + exp(\theta)} 
\end{align}

After that fair amount of algebra, let's rewrite \[1 - p\] as
\[(1+exp(\theta))^{-1}\] because a negative exponent denotes a fraction.
This step is really important because of \textbf{prelimiary 4}. Thus,
inputting \(nlog(1+exp(\theta))^{-1}\) back in -\\
\hspace*{0.333em}\\
We can rewrite, now: ~

10a. Input \(nlog(1+exp(\theta))^{-1}\) in place of \(nlog(1-p)\).

~

10b. Input \(-nlog(1+exp(\theta))\) because \(log(a^4) = 4log(a)\)

~

\begin{align}
\tag{10a}
f(x, \theta, \phi)=y(\theta) + nlog(1+exp(\theta))^{-1} + log{n\choose y} 
\end{align}

\begin{align}
\tag{10b}
f(x, \theta, \phi)=y(\theta) - nlog(1+exp(\theta)) + log{n\choose y}
\end{align}

Believe it or not, this is now in exponential form, where:

\[y(\theta) = ylog(\frac{p}{1-p})\] \[b(\theta) = nlog(1+exp(\theta))\]
\[c(y, \phi) = log{n\choose y}\] \[a(\phi)=1\]
\[f(x, \theta, \phi)=exp([y(\theta) - nlog(1+exp(\theta))] + log{n\choose y})\]

\hypertarget{finding-expectations---a-little-bit-of-calculus-optional}{%
\subsection{Finding Expectations - a little bit of calculus
(optional):}\label{finding-expectations---a-little-bit-of-calculus-optional}}

It turns out, that the expectation of a distribution in the exponential
form is the first derivative of \(b(\theta)\) and the variance is
\(a(\phi)\) times the second derivative of \(b(\theta)\) 1.
\(\mu = b'(\theta)\) 2. \(\sigma^2 = b''(\theta)*a(\phi)\)

\hypertarget{expectationmean}{%
\subsubsection{Expectation/mean}\label{expectationmean}}

\begin{align}
=b'(\theta) = n\biggl[log(1+exp(\theta))\biggr] \\
=n\bigg[exp(\theta)*\frac{1}{1+exp(\theta)}\bigg] \\
=n\bigg[\frac{exp(\theta)}{1+exp(\theta)}\bigg] \\
= n*p =np
\end{align}

This is great news. The first derivative worked out and we found the
mean. This is what we need to pass through the link function,
eventually.

\hypertarget{variance-optional}{%
\subsubsection{Variance (optional)}\label{variance-optional}}

Given the definition above, this is how I solved. I use the chain and
product rule instead of the quotient rule\ldots sue me.

\begin{align}
a(\phi)b''(\theta) = 1*n\biggl[\frac{exp(\theta)}{1+exp(\theta)}\biggr] \\ 
= n\biggl[exp({\theta})*(1+exp(\theta)^{-1})\biggr] + \biggl[exp(\theta)*-1*exp(\theta)*
(exp(1+\theta)^{-2})\biggr] \\
=n[\frac{exp(\theta)}{{1+exp(theta)}} - \frac{exp(2\theta)}{(exp(1+\theta)^{2})}] \\
=n\frac{exp(\theta)}{1+exp(\theta)}(1-exp(\theta)) \\
=npq
\end{align}

If you know the mean of the binomial, you'll realize this isn't super
important to go through all this work, but it's good to see.

\hypertarget{linking-mu-and-theta}{%
\subsubsection{\texorpdfstring{Linking \(\mu\) and
\(\theta\)}{Linking \textbackslash mu and \textbackslash theta}}\label{linking-mu-and-theta}}

So the key aspect of a link function is how we go from \(\mu\) to
\(\theta\). That is, \[ f(\mu) = \theta\] is effectively the link
function.

\hypertarget{finding-the-link-function}{%
\subsection{Finding the link function}\label{finding-the-link-function}}

If you skipped the calculus portion, all we need to know now is \(\mu\)
in terms of \(\theta\)

\[\mu = n*p = n*\biggr(\frac{exp(\theta)}{1+\exp(\theta)}\biggr)\] We
need to solve for \(\theta\). We'll start by multiplying both sides of
the equation by \(1+exp(\theta)\) and then follow the same process as
\texttt{step\ 9} but replace \texttt{p} with \(n*exp(\theta)\)

\begin{align}
\mu(1+exp(\theta)) = n*exp(\theta) \\
\mu+(\mu*\exp(\theta)) = n*exp(\theta) \\
\mu = n*exp(\theta) - \mu*exp(\theta) \\
\mu = exp(\theta)(n-\mu) \\ 
\frac{\mu}{n-\mu}=exp(\theta) \\
\log(\frac{\mu}{n-\mu}) = \theta
\end{align}

And that's your link function! But note, \(\mu = n*p\) so we can rewrite
it as:

\[\theta = log(\frac{n*p}{n-(n*p))})\]

Which, either factoring out n/n, or by substituting 1 for n as is the
case for bernoulli trials like in scenarios we're used to with logistic
regression:

\[\theta = log(\frac{1*p}{1-(1*p))})\] \[\theta = log(\frac{p}{1-p})\]
And, now this looks awfully like logistic regression.
=======
\[f(x, \theta, \phi) = exp[\frac{y\theta-b(\theta)}{a(\phi)} + c(y, \phi)]\]

One notes that this looks nothing like the binomial distribution in its
generalized linear model form (glm)

\[f(y) = {n \choose y}p^y(1-p)^{n-y}\] ..

So we'll have to transform it into the form above.
>>>>>>> 19bdb9ffa4f78b31e460bde0aaae9e85ef799363

\end{document}
